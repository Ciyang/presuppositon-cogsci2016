% Presupposition report (winter 2015)

In this section, we will briefly review the main ideas and motivations of the
 pragmatic approach to projection.

There are many reasons why one might prefer a pragmatic approach to a 
 semantic one. 
First, change-of-state verbs systematically show projective behavior. 
Therefore, a generalization would be missing if their projective contents are
 lexically-encoded properties unrelated to each other.
Second, projection interacts with the contextual question under discussion,
 as can be seen from the following example \cite{Geurts1995:Presupposing}.
Imagine that Bob asked Alice: ``I notice that John keeps chewing on his pencil. Did he recently stop smoking?''
In this context Bob is interested in whether there was a change from smoking to non-smoking, rather than simply whether John is currently a smoker. 
As a result, he would not infer from an answer of ``no (it's just his habbit)'' that 
 John smoked in the past. 
Third, projection is sensitive to focus.

Such evidence suggests that projection is probably not purely semantic, but how do we 
 capture it using general conversational principles?
For example, when Alice says ``John did not stop smoking,'' technically it is true 
 even if John has never smoked. 
Then how can Bob infer that John smoked in the past when Alice uses it in reply to his question whether John smokes?

We will sketch the intuitive ideas here, and formalize them with a probabilistic model in the next section. 
Bob's reasoning goes roughly as follows. 
Taken literally, Alice's utterance seems under-informative because whether or not
 John smokes is compatible with what she said.
However, given that she is assumed to know whether John smokes, be cooperative, and
 be able to directly address the question under discussion, 
 she should not have said something under-informative such as ``John did not stop smoking.''
But she did. 
So perhaps her answer is informative after all.
But how?
Maybe she has taken some additional information for granted, assuming that it is in 
 the \emph{common ground} between the speaker and the listener.
What might that information be?
Well, if Alice took for granted that John smoked in the past, then, 
  together with ``John did not stop smoking,'' this information would mean that 
 John still smokes, which fully answers the QUD whether John smokes now.
In other words, assuming that Alice took for granted that John smoked in the past 
 helps best explain Alice's choice of her utterance. 
Therefore, Bob the listener would infer that John smoked in the past, which is why 
 this information survives under negation.

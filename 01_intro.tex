% Presupposition cogsci 2016
Natural language gives rise to various kinds of inferences. 
An important class of such inferences is \emph{entailment}. 
For example, from the sentence \emph{John danced} we can infer 
 that John moved. 
When a sentence is embedded under certain operators, such as negation, 
 its entailments may no longer hold. 
For the previous example, the negation of \emph{John danced}, i.e., 
 \emph{John did not dance}, no longer entails that John moved.
This illustrates the well-known fact that negation is an
 \emph{entailment-canceling} operator.

However, certain inferences seem to be able to survive entailment-canceling operators.
For instance, think of a scenario in which Alice and Bob are talking about John, 
 an old friend of Alice's who is visiting her. 
Bob has never met John before so he knows nothing about him. 
Bob asks Alice, ``Does John smoke?'' 
Alice replies, ``John stopped smoking.'' 
At this point, what Bob infers about John from Alice's statement is twofold: 
 (i) that John does not smoke now, and (ii) that John smoked in the past.%
\footnote{Alice's answer has another reading, which is about a particular smoking event 
 (e.g., John lit up a cigarette and finished it).
We only discuss the habitual reading in this paper but our proposed analysis can be similarly 
 applied to the event reading.}
 
Both (i) and (ii) are highly plausible inferences and are arguably entailments of
 the sentence ``John stopped smoking.''
But what happens when this sentence is embedded under negation? 
That is, imagine another scenario which is identical to the previous one except that 
 Alice's reply is ``John did not stop smoking.'' 
What would Bob infer in this case?
Intuitively, it seems that Bob would infer (i$'$) that John smokes now, 
 and (ii$'$) that John smoked in the past.
 
Inferences (i) and (i$'$) are opposite.
This is expected, since negation is an entailment-canceling operator.
However, (ii) and (ii$'$) are exactly the same.
This is surprising and puzzling: given that negation is entailment-canceling, 
 why and how does inference (ii) survive?
 
In natural language semantics and pragmatics, inference (ii) is called the
 \emph{projective content} under negation of the sentence ``John stopped smoking.'' 
Our scenario is an example showing that a change-of-state verb has 
 the information about the past as its projective content under negation. 
There are many other types of projective contents 
 (e.g., the complement of \emph{know}) under different operators 
 (e.g., questions modals) identified and discussed in the literature. 
And the problem of explaining how certain inferences such as (ii) can
 survive entailment-canceling operators is called the \emph{projection problem}.

There are two main approaches to the projection problem.
According to the \emph{semantic} approach, projective contents are conventional properties of
 lexical items \cite<e.g.,>{frege1948sense,heimkratzer1998}. 
According to the \emph{pragmatic} approach, projection can be
 derived from general conversational principles 
 \cite<e.g.,>{Stalnaker1974:Pragmatic-Presuppositions,Simons2001:On-the-Conversational,simons2006foundational}. 
 
In this paper, based on previous ideas of the pragmatic approach, we propose 
 an extension to Rational Speech-Act (RSA) model
 \cite{FrankGoodman2012:Predicting-Pragmatic-Reasoning-,GoodmanStuhlmuller2013:Knowledge-and-I} to account for the projection phenomena of change-of-state verbs 
 under negation.
The model correctly predicts the projective behavior in our example scenario. 
And it further predicts an interaction between projective behavior and the  
 \emph{question under discussion} (QUD) \cite{Roberts2012:Information-Structure} for change-of-state verbs under negation.

The rest of the paper is as follows. We first briefly review the basic ideas of 
 the pragmatic approach to projection. Next, we introduce an extension to the 
 standard RSA model incrementally, pointing out each additional assumption needed and the motivation.
Finally we further discuss the full model's predictions and implications and suggest future directions.



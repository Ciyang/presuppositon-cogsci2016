% Presupposition cogsci 2016


\begin{quotation}
``When did you stop beating your wife?'' ---comedian
\end{quotation}

From \emph{John danced} we can infer \emph{John moved}, but we cannot infer this from \emph{John didn't dance}---the inference that \emph{John moved} is canceled by negation.
In contrast, from both \emph{John stopped smoking} and \emph{John didn't stop smoking} we are likely to infer that \emph{John used to smoke}---this inference is said to \emph{project} over negation.
In natural language semantics and pragmatics, an inference that survives an operator that usually cancels inferences is called
 \emph{projective content} of the sentence under that operator. 
Change-of-state verbs can have information about the past as projective content under negation: \emph{John used to smoke} is projective content under negation of ``John stopped smoking.'' 
There are many other types of projective contents 
 (e.g., the complement of \emph{know}) under different operators 
 (e.g., questions, modals) identified and discussed in the literature. 
The problem of explaining how certain inferences can
 survive entailment-canceling operators is called the \emph{projection problem}.



%Natural language gives rise to various kinds of inferences. 
%One important class of such inferences is \emph{entailment}. 
%For example, from the sentence \emph{John danced} we can infer 
% that John moved. 
%When a sentence is embedded under certain operators, such as negation, 
% its entailments may no longer hold. 
%For the previous example, the negation of \emph{John danced}, i.e., 
% \emph{John did not dance}, no longer entails that John moved.
%This illustrates the well-known fact that negation is an
% \emph{entailment-canceling} operator.

%However, certain inferences seem to be able to survive entailment-canceling operators.
%For instance, think of a scenario in which Alice and Bob are talking about John, 
% an old friend of Alice's who is visiting her. 
%Bob has never met John before so he knows nothing about him. 
%Bob asks Alice, ``Does John smoke?'' 
%Alice replies, ``John stopped smoking.'' 
%At this point, what Bob infers about John from Alice's statement is twofold: 
% (i) that John does not smoke now, and (ii) that John smoked in the past.%
%\footnote{Alice's answer has another reading, which is about a particular smoking event 
% (e.g., John lit up a cigarette and finished it).
%We only discuss the habitual reading in this paper but our proposed analysis can be similarly 
% applied to the event reading.}
% 
%Both (i) and (ii) are highly plausible inferences and are arguably entailments of
% the sentence ``John stopped smoking.''
%But what happens when this sentence is embedded under negation? 
%That is, imagine another scenario which is identical to the previous one except that 
% Alice's reply is ``John did not stop smoking.'' 
%What would Bob infer in this case?
%Intuitively, it seems that Bob would infer (i$'$) that John smokes now, 
% and (ii$'$) that John smoked in the past.
% 
%Inferences (i) and (i$'$) are opposite.
%This is expected, since negation is an entailment-canceling operator.
%However, (ii) and (ii$'$) are exactly the same.
%This is surprising and puzzling: given that negation is entailment-canceling, 
% why and how does inference (ii) survive?
 

There are two main approaches to the projection problem.
According to the \emph{semantic} approach, projective contents are conventional properties of
 lexical items \cite<e.g.,>{frege1948sense,heimkratzer1998}. 
According to the \emph{pragmatic} approach, projection can be
 derived from general conversational principles 
 \cite<e.g.,>{Stalnaker1974:Pragmatic-Presuppositions,Simons2001:On-the-Conversational,simons2006foundational}. 
In this paper, we build on and formalize previous ideas of the pragmatic approach.
We extend the probabilistic Rational Speech Act (RSA) model
 \cite{FrankGoodman2012:Predicting-Pragmatic-Reasoning-,GoodmanStuhlmuller2013:Knowledge-and-I} to account for the projection phenomena of change-of-state verbs 
 under negation.
 \ndg{need to say a bit more about the model: why and what are the basic ideas?}
The model correctly predicts the projective behavior in our example scenario. 
And it further predicts an interaction between projective behavior and the  
 \emph{question under discussion} (QUD) \cite{Roberts2012:Information-Structure} for change-of-state verbs under negation.

\ndg{somewhere we need to cite / discuss Degen, Tessler, Goodman 2015}

The rest of the paper is as follows. We first briefly review the basic ideas of the pragmatic approach to projection. 
Next, we describe a sequence of extensions to the standard RSA model that ultimately predict the correct pattern of projection. 
%pointing out each additional assumption needed and the motivation.
Finally we discuss the full model's predictions, implications and possible future directions.


